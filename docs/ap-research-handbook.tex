\documentclass[]{book}
\usepackage{lmodern}
\usepackage{amssymb,amsmath}
\usepackage{ifxetex,ifluatex}
\usepackage{fixltx2e} % provides \textsubscript
\ifnum 0\ifxetex 1\fi\ifluatex 1\fi=0 % if pdftex
  \usepackage[T1]{fontenc}
  \usepackage[utf8]{inputenc}
\else % if luatex or xelatex
  \ifxetex
    \usepackage{mathspec}
  \else
    \usepackage{fontspec}
  \fi
  \defaultfontfeatures{Ligatures=TeX,Scale=MatchLowercase}
\fi
% use upquote if available, for straight quotes in verbatim environments
\IfFileExists{upquote.sty}{\usepackage{upquote}}{}
% use microtype if available
\IfFileExists{microtype.sty}{%
\usepackage{microtype}
\UseMicrotypeSet[protrusion]{basicmath} % disable protrusion for tt fonts
}{}
\usepackage[margin=1in]{geometry}
\usepackage{hyperref}
\hypersetup{unicode=true,
            pdftitle={AP Research Handbook},
            pdfborder={0 0 0},
            breaklinks=true}
\urlstyle{same}  % don't use monospace font for urls
\usepackage{natbib}
\bibliographystyle{apalike}
\usepackage{color}
\usepackage{fancyvrb}
\newcommand{\VerbBar}{|}
\newcommand{\VERB}{\Verb[commandchars=\\\{\}]}
\DefineVerbatimEnvironment{Highlighting}{Verbatim}{commandchars=\\\{\}}
% Add ',fontsize=\small' for more characters per line
\usepackage{framed}
\definecolor{shadecolor}{RGB}{248,248,248}
\newenvironment{Shaded}{\begin{snugshade}}{\end{snugshade}}
\newcommand{\AlertTok}[1]{\textcolor[rgb]{0.94,0.16,0.16}{#1}}
\newcommand{\AnnotationTok}[1]{\textcolor[rgb]{0.56,0.35,0.01}{\textbf{\textit{#1}}}}
\newcommand{\AttributeTok}[1]{\textcolor[rgb]{0.77,0.63,0.00}{#1}}
\newcommand{\BaseNTok}[1]{\textcolor[rgb]{0.00,0.00,0.81}{#1}}
\newcommand{\BuiltInTok}[1]{#1}
\newcommand{\CharTok}[1]{\textcolor[rgb]{0.31,0.60,0.02}{#1}}
\newcommand{\CommentTok}[1]{\textcolor[rgb]{0.56,0.35,0.01}{\textit{#1}}}
\newcommand{\CommentVarTok}[1]{\textcolor[rgb]{0.56,0.35,0.01}{\textbf{\textit{#1}}}}
\newcommand{\ConstantTok}[1]{\textcolor[rgb]{0.00,0.00,0.00}{#1}}
\newcommand{\ControlFlowTok}[1]{\textcolor[rgb]{0.13,0.29,0.53}{\textbf{#1}}}
\newcommand{\DataTypeTok}[1]{\textcolor[rgb]{0.13,0.29,0.53}{#1}}
\newcommand{\DecValTok}[1]{\textcolor[rgb]{0.00,0.00,0.81}{#1}}
\newcommand{\DocumentationTok}[1]{\textcolor[rgb]{0.56,0.35,0.01}{\textbf{\textit{#1}}}}
\newcommand{\ErrorTok}[1]{\textcolor[rgb]{0.64,0.00,0.00}{\textbf{#1}}}
\newcommand{\ExtensionTok}[1]{#1}
\newcommand{\FloatTok}[1]{\textcolor[rgb]{0.00,0.00,0.81}{#1}}
\newcommand{\FunctionTok}[1]{\textcolor[rgb]{0.00,0.00,0.00}{#1}}
\newcommand{\ImportTok}[1]{#1}
\newcommand{\InformationTok}[1]{\textcolor[rgb]{0.56,0.35,0.01}{\textbf{\textit{#1}}}}
\newcommand{\KeywordTok}[1]{\textcolor[rgb]{0.13,0.29,0.53}{\textbf{#1}}}
\newcommand{\NormalTok}[1]{#1}
\newcommand{\OperatorTok}[1]{\textcolor[rgb]{0.81,0.36,0.00}{\textbf{#1}}}
\newcommand{\OtherTok}[1]{\textcolor[rgb]{0.56,0.35,0.01}{#1}}
\newcommand{\PreprocessorTok}[1]{\textcolor[rgb]{0.56,0.35,0.01}{\textit{#1}}}
\newcommand{\RegionMarkerTok}[1]{#1}
\newcommand{\SpecialCharTok}[1]{\textcolor[rgb]{0.00,0.00,0.00}{#1}}
\newcommand{\SpecialStringTok}[1]{\textcolor[rgb]{0.31,0.60,0.02}{#1}}
\newcommand{\StringTok}[1]{\textcolor[rgb]{0.31,0.60,0.02}{#1}}
\newcommand{\VariableTok}[1]{\textcolor[rgb]{0.00,0.00,0.00}{#1}}
\newcommand{\VerbatimStringTok}[1]{\textcolor[rgb]{0.31,0.60,0.02}{#1}}
\newcommand{\WarningTok}[1]{\textcolor[rgb]{0.56,0.35,0.01}{\textbf{\textit{#1}}}}
\usepackage{longtable,booktabs}
\usepackage{graphicx,grffile}
\makeatletter
\def\maxwidth{\ifdim\Gin@nat@width>\linewidth\linewidth\else\Gin@nat@width\fi}
\def\maxheight{\ifdim\Gin@nat@height>\textheight\textheight\else\Gin@nat@height\fi}
\makeatother
% Scale images if necessary, so that they will not overflow the page
% margins by default, and it is still possible to overwrite the defaults
% using explicit options in \includegraphics[width, height, ...]{}
\setkeys{Gin}{width=\maxwidth,height=\maxheight,keepaspectratio}
\IfFileExists{parskip.sty}{%
\usepackage{parskip}
}{% else
\setlength{\parindent}{0pt}
\setlength{\parskip}{6pt plus 2pt minus 1pt}
}
\setlength{\emergencystretch}{3em}  % prevent overfull lines
\providecommand{\tightlist}{%
  \setlength{\itemsep}{0pt}\setlength{\parskip}{0pt}}
\setcounter{secnumdepth}{5}
% Redefines (sub)paragraphs to behave more like sections
\ifx\paragraph\undefined\else
\let\oldparagraph\paragraph
\renewcommand{\paragraph}[1]{\oldparagraph{#1}\mbox{}}
\fi
\ifx\subparagraph\undefined\else
\let\oldsubparagraph\subparagraph
\renewcommand{\subparagraph}[1]{\oldsubparagraph{#1}\mbox{}}
\fi

%%% Use protect on footnotes to avoid problems with footnotes in titles
\let\rmarkdownfootnote\footnote%
\def\footnote{\protect\rmarkdownfootnote}

%%% Change title format to be more compact
\usepackage{titling}

% Create subtitle command for use in maketitle
\providecommand{\subtitle}[1]{
  \posttitle{
    \begin{center}\large#1\end{center}
    }
}

\setlength{\droptitle}{-2em}

  \title{AP Research Handbook}
    \pretitle{\vspace{\droptitle}\centering\huge}
  \posttitle{\par}
    \author{}
    \preauthor{}\postauthor{}
      \predate{\centering\large\emph}
  \postdate{\par}
    \date{2019-05-29}

\usepackage{booktabs}
\usepackage{amsthm}
\makeatletter
\def\thm@space@setup{%
  \thm@preskip=8pt plus 2pt minus 4pt
  \thm@postskip=\thm@preskip
}
\makeatother

%Allow for more levels in bullet lists
%https://github.com/Witiko/markdown/issues/2
%https://tex.stackexchange.com/questions/41408/a-five-level-deep-list
\usepackage{enumitem}
\setlistdepth{20}
\renewlist{itemize}{itemize}{20}
% initially, use dots for all levels
\setlist[itemize]{label=$\cdot$}

% customize the first 3 levels
\setlist[itemize,1]{label=\textbullet}
\setlist[itemize,2]{label=--}
\setlist[itemize,3]{label=*}

\begin{document}
\maketitle

{
\setcounter{tocdepth}{1}
\tableofcontents
}
\hypertarget{research-philosophy-ethics}{%
\chapter*{Research Philosophy \& Ethics}\label{research-philosophy-ethics}}
\addcontentsline{toc}{chapter}{Research Philosophy \& Ethics}

\begin{itemize}
\tightlist
\item
  No Plagiarism
\item
  Institutional Review Board
\item
  Data privacy standards
\end{itemize}

This AP Research Handbook is still a work in progress. Please excuse any blank sections and filler text.

\hypertarget{intro}{%
\chapter{Research Question}\label{intro}}

You can label chapter and section titles using \texttt{\{\#label\}} after them, e.g., we can reference Chapter \ref{intro}. If you do not manually label them, there will be automatic labels anyway, e.g., Chapter \ref{qualmethods}.

Figures and tables with captions will be placed in \texttt{figure} and \texttt{table} environments, respectively.

\hypertarget{literature-review}{%
\chapter{Literature Review}\label{literature-review}}

List resources for conducting literature review.
Show example of literature review with inline citations.
Show ways to keep track of sources for bibliography.

\begin{itemize}
\tightlist
\item
  \href{https://writing.wisc.edu/handbook/assignments/reviewofliterature/}{How to Write a Literature Review}

  \begin{itemize}
  \tightlist
  \item
    contains example literature reviews from political science, philosophy, and chemistry.
  \end{itemize}
\end{itemize}

Consider using a reference management system like Mendeley to organize your sources as you conduct your literature review. In fact, Mendeley has a Literature Search function, so you can manage sources and conduct literature reviews at the same time. See the Bibliography Management Section for more information on managing sources.

\begin{itemize}
\item
  Databases for Literature Reviews

  \begin{itemize}
  \tightlist
  \item
    \href{https://doaj.org/subjects}{Directory of Open Access Journals}

    \begin{itemize}
    \tightlist
    \item
      Browse by subjects in the humanities and sciences. This can be your starting point if you have not developed a research topic.
    \end{itemize}
  \item
    \href{https://arxiv.org/}{arXiv}

    \begin{itemize}
    \tightlist
    \item
      Open-access journal articles in fields such as mathematics, statistics, economics, physics, quantitative biology, quantiative finance, and electrical engineering
    \item
      \href{https://arxiv2bibtex.org}{arXiv to BibTex}: Outputs automated citations in BibTeX and other formats by typing the arXiv number of the article. For instance, just type in 1905.03758 into the search engine if the article is labeled arXiv: 1905.03758.
    \item
      Alternatively, use \href{https://www.mendeley.com/reference-management/web-importer\#id_2}{Mendeley Web Importer} to import article into Mendeley Desktop for automated citation outputs.
    \end{itemize}
  \item
    \href{https://blog.mendeley.com/2013/07/08/new-release-literature-search-from-within-mendeley-deskop/}{Mendeley Literature Search}

    \begin{itemize}
    \tightlist
    \item
      Download \href{https://www.mendeley.com/download-desktop/}{Mendeley Desktop} and register for a free account. Mendeley Desktop syncs with your online Mendeley account, but the literature search is currently only available in the desktop version.
    \item
      Mendeley is primarly a reference managements software, so you can organize your citations as you conduct your literature review.
    \end{itemize}
  \item
    \href{https://core.ac.uk/}{CORE}

    \begin{itemize}
    \tightlist
    \item
      Search engine with the world's largest collectin of open-access research papers.
    \item
      For batch searches of metadata and full texts, you may consider requesting a free API key to use the \href{https://core.ac.uk/services/api/}{Core API}.
    \end{itemize}
  \item
    \href{https://www.scienceopen.com}{ScienceOpen}

    \begin{itemize}
    \tightlist
    \item
      Search for content, authors, collections, and journals in the \href{https://www.scienceopen.com/search\#advanced}{advanced search}, where you have the option to search by discipline or key word.
    \end{itemize}
  \item
    \href{https://app.dimensions.ai/discover/publication}{Dimensions}

    \begin{itemize}
    \tightlist
    \item
      Search for articles in clincial sciences, biochemistry, public health, physical chemistry, and materials engineering.
    \end{itemize}
  \item
    \href{https://www.ebsco.com/open-access}{EBSCO Open Access}

    \begin{itemize}
    \tightlist
    \item
      Search open-access journals and dissertations. Note that dissertations can vary in quality, since they have not gone through peer review.
    \item
      AP Research students should have access to a free EBSCO account from the AP Capstone program.
    \end{itemize}
  \item
    \href{https://papers.ssrn.com/sol3/DisplayJournalBrowse.cfm}{SSRN}

    \begin{itemize}
    \tightlist
    \item
      Many of the social science articles are free access.
    \end{itemize}
  \item
    \href{https://eric.ed.gov/}{ERIC: Institute of Educadtion Sciences}

    \begin{itemize}
    \tightlist
    \item
      Search for articles related to to education research.
    \item
      The search engine includes the open to search for full-text articles.
    \end{itemize}
  \item
    \href{https://dblp.org/}{dblp: Computer Science Bibliography}

    \begin{itemize}
    \tightlist
    \item
      Index of major computer science publications.
    \item
      Option to search for open-access articles.
    \end{itemize}
  \item
    \href{https://www.econbiz.de/}{EconBiz}

    \begin{itemize}
    \tightlist
    \item
      Search for journal articles, working papers, and conference papers in economics and business.
    \item
      Option to search for open-access articles.
    \end{itemize}
  \item
    \href{https://support.jstor.org/hc/en-us/articles/115004760028-MyJSTOR-How-to-Register-Get-Free-Access-to-Content}{MyJSTOR}

    \begin{itemize}
    \tightlist
    \item
      You can sign up for a free MyJSTOR account to access up to six articles a month for free.
    \item
      This may be helpful for accessing articles that are not open access.
    \end{itemize}
  \end{itemize}
\item
  Tips for Accessing Paywalled Articles

  \begin{itemize}
  \tightlist
  \item
    Search for the author's website. Many researchers have draft manuscripts on their websites or research profiles on sites such as \href{https://www.researchgate.net/}{ResearchGate}.
  \item
    Consult your school's research librarian for other ways to access the article.
  \item
    Send the author an e-mail to request for a digital copy of the article. You should provide context in the e-mail request by including a brief description of your AP Research project and its relevance and connection to the author's article.
  \end{itemize}
\end{itemize}

\hypertarget{bibliography-management}{%
\chapter{Bibliography Management}\label{bibliography-management}}

While the bibliography is placed at the end of research papers, reference management begins as soon as you begin your literature review.

\begin{itemize}
\item
  \href{https://guides.library.yale.edu/bibtex/bibstyles}{Managing Citations in LaTeX}
\item
  \href{https://www.mendeley.com/download-desktop/}{Mendeley Desktop Download}

  \begin{itemize}
  \tightlist
  \item
    The download will prompt you to create a free account. Mendeley Desktop is synced with your online account.
  \end{itemize}
\item
  \href{https://www.mendeley.com/reference-management/web-importer\#id_2}{Mendeley Web Importer}

  \begin{itemize}
  \tightlist
  \item
    As you search for research articles online, you can use the Mendeley Web Importer to import citations into your Mendeley Desktop. If the importer can recognize the online article's metadata, it will automatically populate the citation entries. If not, you can still enter the citation entries manually and import into the Mendeley Desktop to keep track of your sources.
  \end{itemize}
\item
  \href{https://guides.library.yale.edu/mendeleybasics/mendeley/mendeley-desktop}{Mendeley Tutorials}
\item
  Exporting .bib files from Mendeley Desktop
\item
  Install MS Word plugin
\item
  \href{https://onhavingwords.wordpress.com/2013/03/19/mendeley-lyx/}{Import Mendelay sources into LyX}
\item
  \href{https://libguides.nus.edu.sg/c.php?g=145734\&p=1433095}{Import .RIS Files into Mendeley}
\end{itemize}

\hypertarget{paper-guidelines}{%
\chapter{Paper Guidelines}\label{paper-guidelines}}

\begin{itemize}
\tightlist
\item
  \href{https://www.overleaf.com/read/ncvzdvcvfxdx}{AP Research Proposal Guidelines}
\item
  \href{https://www.overleaf.com/read/hxctjpdnvffw}{AP Research Paper Guidelines (LaTeX)}

  \begin{itemize}
  \tightlist
  \item
    draft in progress
  \end{itemize}
\end{itemize}

\hypertarget{file-organization}{%
\chapter{File Organization}\label{file-organization}}

\hypertarget{research-concepts}{%
\chapter{Research Concepts}\label{research-concepts}}

\hypertarget{reliability-validity}{%
\section{Reliability \& Validity}\label{reliability-validity}}

\hypertarget{accuracy-vs.-precision}{%
\section{Accuracy vs.~Precision}\label{accuracy-vs.-precision}}

\hypertarget{bias-vs.-variance-tradeoff}{%
\section{Bias vs.~Variance Tradeoff}\label{bias-vs.-variance-tradeoff}}

\hypertarget{curse-of-dimensionality}{%
\section{Curse of Dimensionality}\label{curse-of-dimensionality}}

\hypertarget{correlation-vs.-causation}{%
\section{Correlation vs.~Causation}\label{correlation-vs.-causation}}

\hypertarget{research-design}{%
\chapter{Research Design}\label{research-design}}

Before you decide how you will conduct your research, read through the \href{https://libguides.usc.edu/writingguide/researchdesigns}{list of research designs} in the USC library guide.

\begin{itemize}
\tightlist
\item
  \href{https://www.nyu.edu/classes/bkg/methods/005847ch1.pdf}{What is Research Design?}
\end{itemize}

\hypertarget{action}{%
\section{Action}\label{action}}

\hypertarget{case-study}{%
\section{Case Study}\label{case-study}}

\hypertarget{causal}{%
\section{Causal}\label{causal}}

\hypertarget{cohort}{%
\section{Cohort}\label{cohort}}

\hypertarget{cross-sectional}{%
\section{Cross Sectional}\label{cross-sectional}}

\hypertarget{descriptive}{%
\section{Descriptive}\label{descriptive}}

\hypertarget{experimental}{%
\section{Experimental}\label{experimental}}

\hypertarget{exploratory}{%
\section{Exploratory}\label{exploratory}}

\hypertarget{historical}{%
\section{Historical}\label{historical}}

\hypertarget{longitudinal}{%
\section{Longitudinal}\label{longitudinal}}

\hypertarget{meta-analysis}{%
\section{Meta-Analysis}\label{meta-analysis}}

\hypertarget{mixed-methods}{%
\section{Mixed Methods}\label{mixed-methods}}

\hypertarget{observational}{%
\section{Observational}\label{observational}}

\hypertarget{philosophical}{%
\section{Philosophical}\label{philosophical}}

\hypertarget{sequential}{%
\section{Sequential}\label{sequential}}

\hypertarget{systematic-review}{%
\section{Systematic Review}\label{systematic-review}}

\hypertarget{quasi-experimental}{%
\section{Quasi-experimental}\label{quasi-experimental}}

\hypertarget{qualmethods}{%
\chapter{Qualitative Research Methods}\label{qualmethods}}

\begin{itemize}
\tightlist
\item
  \href{https://course.ccs.neu.edu/is4800sp12/resources/qualmethods.pdf}{Qualitative Research Methods Field Guide}
\end{itemize}

\hypertarget{case-study-1}{%
\section{Case Study}\label{case-study-1}}

\hypertarget{narrative}{%
\section{Narrative}\label{narrative}}

\hypertarget{phenomenological}{%
\section{Phenomenological}\label{phenomenological}}

\hypertarget{ethnography}{%
\section{Ethnography}\label{ethnography}}

\hypertarget{grounded-theory}{%
\section{Grounded Theory}\label{grounded-theory}}

\begin{itemize}
\tightlist
\item
  \href{https://pdfs.semanticscholar.org/5886/43f9ded159acc42daeefed6f1d1952bea546.pdf}{Grounded Theory as Scientific Method}
\end{itemize}

\hypertarget{quantitative-methods}{%
\chapter{Quantitative Methods}\label{quantitative-methods}}

\hypertarget{causal-inference}{%
\section*{Causal Inference}\label{causal-inference}}
\addcontentsline{toc}{section}{Causal Inference}

These notes are based on Professor Masten's \href{https://modu.ssri.duke.edu/topic/introduction-causal-inference}{online course} on Causal Inference at the Social Science Research Institute at Duke.

\begin{itemize}
\item
  Causal effect is often easy to detect with simple actions for which the effect immediately follows (e.g., you caused the alarm clock to stop ringing by pressing the snooze button)
\item
  With multiple causes and delayed effects, causality is much harder to detect.
\item
  Measurement:

  \begin{itemize}
  \tightlist
  \item
    Unit of analysis: countries, city blocks, people, firms, etc.
  \item
    Outcome variable: the characteristic of the unit of analysis that we want to affect
  \item
    Policy/treatment variable: the characteristic that we use to change the outcome varialbe
  \end{itemize}
\item
  A lot of characteristics cannot readily be quantified, so we often use proxy variables. For example, GDP could be a proxy for economic development.
\item
  Causality: how an intervention in the policy variable affects the outcome variable
\item
  Data:

  \begin{itemize}
  \tightlist
  \item
    The value of the policy variable has to vary in the dataset. Without this variation, you can't analyze how changes in the policy variable might affect the outcome variable.
  \item
    Larger standard deviation = larger variation
  \end{itemize}
\item
  Correlation vs.~Causation

  \begin{itemize}
  \tightlist
  \item
    If the policy and outcome variables are correlated, this does not necessarily imply a casual relationship.
  \item
    Selection Problem: when units get to choose their policy variable, correlations between policy and outcome variables are unlikely to be causal.

    \begin{itemize}
    \tightlist
    \item
      Example: Neighborhoods with a lot of trees tend to have less crime.
    \item
      If this were a casual relationship, then we could plant more trees in a neighborhood and expect crime to go down. However, this is unlikely. More likely, people who tend to commit less crimes chose to live in neighborhoods with tree-lined streets.
    \end{itemize}
  \end{itemize}
\item
  Average Treatment Effect

  \begin{itemize}
  \tightlist
  \item
    Causal effects vary among people, so there is a distribution of causal effects in the population.
  \item
    Theoretical ideal: you would know the unit level of causal effect for each person and thus make individualized treatment decisions. This is impossible in practice. You can't know the effect of receiving and not receiving treatment for an individual.
  \item
    Unit-level causal effect: difference in outcome between treatment \& control, holding all other variables fixed
  \item
    Avg. treatment effect (ATE): avg. of all values for unit-level causal effects in a population
  \item
    Avg. outcome under the policy: avg. outcome when everyone is affected by the policy (i.e., receives treatment)
  \item
    Avg. outcome without the policy: avg. outcome when everyone is not affected by policy (i.e., does not receive treatment)
  \item
    ATE = Avg. outcome under policy - Avg. outcome w/o policy
  \end{itemize}
\end{itemize}

\hypertarget{experiments}{%
\subsection{Experiments}\label{experiments}}

\begin{itemize}
\item
  Controlled Experiments

  \begin{itemize}
  \tightlist
  \item
    Control group does not receive treatment
  \item
    Experimental group receives treatment
  \item
    All possible factors that could affect the outcome are identical for both groups, except for the treatment
  \item
    Difference in the outcome between the two groups is the treatment effect
  \item
    Typically used in hard sciences, but difficult to achieve in social sciences given the myriad of factors, many of which are difficult to measure and control
  \end{itemize}
\item
  Randomized Experiments

  \begin{itemize}
  \tightlist
  \item
    Split units randomly into two large groups: treatment or control
  \item
    Right after randomization and before the experiment, both groups should be similar (i.e., avg. values of factors should be about the same), because the split was done randomly and the groups are very large
  \item
    Since the two groups are similar in all factors except treatment, changes in the \emph{average} outcomes are due to treatment
  \item
    Complications:

    \begin{itemize}
    \tightlist
    \item
      Noncompliance: Even when you randomly assign treatment, people in the treatment group may not all decide to take the treatment. Also, some people in the control group, who should not receive the treatment, might decide to get the treatment.

      \begin{itemize}
      \tightlist
      \item
        Solution 1: Intent to Treat Analysis

        \begin{itemize}
        \tightlist
        \item
          The \emph{intent} to provide treatment is by design random regardless of treatment non-compliance.
        \item
          Thus, we can examine the causal effect of the option of providing treatment.
        \item
          Downside: cannot analyze causal effect of treatment itself
        \item
          For example, in the Oregon Health Experiment, while a lottery randomly selected people to receive free Medicaid, there was noncompliance in both the treatment/control groups. Original interpretation (effect of Medicaid on health outcomes) can be revised to effect of Medicaid lottery assignment on health outcomes.
        \end{itemize}
      \item
        Solution 2: Instrumental Variables

        \begin{itemize}
        \tightlist
        \item
          Advantage: We can analyze the causal effect of the treatment (not just the option of treatment) for a subset of the population.
        \item
          Downside: cannot analyze average treatment effect over the entire population
        \end{itemize}
      \item
        Solution 3: Assume random compliance

        \begin{itemize}
        \tightlist
        \item
          Assume people comply with their treatment assignment.
        \item
          Just drop the entries of the non-compliers.
        \item
          Advantage: We can analyze the causal effect of the treatment over the entire population.
        \item
          Downside: Decision to not comply is probably not random. We don't observe the reasons for non-compliance.
        \end{itemize}
      \item
        Solution 4: Bounds analysis

        \begin{itemize}
        \tightlist
        \item
          Get lower/upper bounds of average treatment effects using extreme scenarios.
        \item
          Upper bounds: assume maximum value for outcome variable for noncompliers
        \item
          Lower bounds: assume minimum value for outcome variable for noncompliers
        \end{itemize}
      \end{itemize}
    \item
      Survey nonresponse

      \begin{itemize}
      \tightlist
      \item
        If nonresponse is not random, you cannot interpret the treatment effect as causal.
      \item
        Example: People in the treatment group with negative outcomes responded to surveys at higher rates than those with positive experiences. Data becomes biased toward negative outcomes for the treatment group.
      \end{itemize}
    \item
      Sample Size: Even with a great research design, small sample size limits statistical inference.
    \item
      Control: You may not be able to control the assignment of treatment.
    \end{itemize}
  \item
    Issues:

    \begin{itemize}
    \tightlist
    \item
      Ethics: Random assignment of treatment may have difficult ethical considerations (e.g., withholding a potentially life-saving drug to a terminally ill patient assigned to a control group in a randomized trial).
    \item
      Extrapolation: It may be hard to extrapolate the results of a randomized experiment to another study if the treatment conditions and features are different.
    \end{itemize}
  \end{itemize}
\item
  Natural Experiments

  \begin{itemize}
  \tightlist
  \item
    Researchers not involved in the research design and data collection in natural experiments, unlike in randomized experiments.
  \item
    observational data used instead
  \item
    Example: charter school lotteries
  \end{itemize}

  \begin{enumerate}
  \def\labelenumi{\arabic{enumi}.}
  \tightlist
  \item
    True Natural Experiments

    \begin{itemize}
    \tightlist
    \item
      treatment was randomly assigned, just not by researcher
    \end{itemize}
  \item
    As-If Natural Experiments

    \begin{itemize}
    \tightlist
    \item
      treatment not actually randomly assigned, but the treatment/control groups appear randomized as though treatment assignment were random)
    \item
      treatment assignment not related to any variables that could affect outcome
    \item
      balance check: characteristics of all observed variables (other than outcome variable) need to be similar between the treatment/control groups

      \begin{itemize}
      \tightlist
      \item
        There could still be differences between groups in unobserved variables.
      \item
        Thus, we cannot prove treatment assignment is truly random, but balanced observed variables between groups would be part of a convincingly arugment that the observational data represents an as-if natural experiment.
      \end{itemize}
    \end{itemize}
  \end{enumerate}
\end{itemize}

\hypertarget{regression-as-causality}{%
\subsection{Regression as Causality}\label{regression-as-causality}}

\hypertarget{instrumental-variables}{%
\subsection{Instrumental Variables}\label{instrumental-variables}}

\hypertarget{stat}{%
\section{Statistical Tests}\label{stat}}

\begin{itemize}
\item
  \href{http://rcompanion.org/handbook/D_03.html}{Choosing a Statistical Test}
\item
  \href{http://www.bmgi.org/sites/bmgi.org/files/HTR\%20MT17.pdf}{Hypothesis Testing Roadmap}
\item
  \href{https://stats.idre.ucla.edu/other/mult-pkg/whatstat/}{Chosing the Correct Statistical Test in SAS, Stata, SPSS, and R}
\item
  \href{http://influentialpoints.com/Training/statistical_mistakes_in_research_use_and_misuse_of_statistics_in_biology.htm}{Uses \& Misuses of Statistics}
\item
  1 group

  \begin{itemize}
  \tightlist
  \item
    interval variables

    \begin{itemize}
    \tightlist
    \item
      1-sample t test for the mean
    \item
      chi-squared test for variance
    \end{itemize}
  \item
    categorical variables

    \begin{itemize}
    \tightlist
    \item
      z test for proportions (2 categories)
    \item
      chi-squared goodness-of-fit
    \end{itemize}
  \item
    ordinal or interval

    \begin{itemize}
    \tightlist
    \item
      one-sample median test
    \end{itemize}
  \end{itemize}
\item
  2 groups (independent groups)

  \begin{itemize}
  \tightlist
  \item
    interval variables

    \begin{itemize}
    \tightlist
    \item
      2 independent sample t-test (equal variances)
    \item
      2 independent sample t-test (unequal variances)
    \item
      F test for difference between 2 variances
    \end{itemize}
  \item
    categorical variables

    \begin{itemize}
    \tightlist
    \item
      z test for difference between 2 proportions
    \item
      chi-squared test for difference between 2 proportions
    \item
      Fisher's exact test
    \end{itemize}
  \end{itemize}
\item
  2 groups (dependent or paired groups)

  \begin{itemize}
  \tightlist
  \item
    paired t-test (interval variables)
  \item
    McNemar's test (categorical variables)
  \item
    Wilcoxon signed ranks test (oridinal or interval variables)
  \end{itemize}
\item
  more than 2 groups (independent groups)

  \begin{itemize}
  \tightlist
  \item
    one-way ANOVA (for interval variables)
  \item
    Kruskal Wallis (for ordinal or interval variables)
  \item
    chi-squared test (for categorical variables)
  \end{itemize}
\item
  more than 2 groups (dependent groups)

  \begin{itemize}
  \tightlist
  \item
    one-way repeated measures ANOVA (for interval variables)
  \item
    repeated measures logistic regression (for categorical variables)
  \item
    Friedman test (for ordinal or interval)
  \end{itemize}
\end{itemize}

\hypertarget{sample-t-test}{%
\subsection{1-sample t-test}\label{sample-t-test}}

\begin{itemize}
\item
  Assumptions:

  \begin{itemize}
  \tightlist
  \item
    data is a simple random sample from population
  \item
    data follows normal distribution
  \item
    by Central Limit Theorm, with sample size \(n>=30\), the sample mean is normally distributed regardless of the population distribution
  \end{itemize}
\item
  Two-tailed Hypothesis:
  \[ H_0: \mu = \mu_0 \]
  \[ H_1: \mu \neq \mu_0 \]
\item
  Test Statistic:
  \[ T = \frac{\overline{X}-\mu_0}{\frac{S}{\sqrt{n}}} \sim t_{(n-1)} \]

  \begin{itemize}
  \tightlist
  \item
    \(\overline{X}\) = sample mean
  \item
    \(\mu_0\) = hypothesized population mean
  \item
    \(S\) = sample standard deviation
  \item
    \(t_{(n-1)}\) = \(t\) distribution with \(n-1\) degrees of freedom
  \end{itemize}
\end{itemize}

\hypertarget{chi-squared-test-for-variance}{%
\subsection{chi-squared test for variance}\label{chi-squared-test-for-variance}}

\hypertarget{z-test-for-proportions}{%
\subsection{z test for proportions}\label{z-test-for-proportions}}

\begin{itemize}
\item
  Assumptions:

  \begin{itemize}
  \tightlist
  \item
    sample proportion \(p = \frac{X}{n}\) comes from random sample in population, where \(X\) is number of events of interest in sample size \(n\).
  \item
    \(p\) follows a binomial distribution, but we can assume normality when \(X\) and \(n-X\) are each at least 5 (old standards) or at least 15 (current standards)
  \end{itemize}
\item
  Two-tailed Hypothesis:
  \[ H_0: \pi = \pi_0 \]
  \[ H_1: \pi \neq \pi_0 \]
\item
  Test Statistic:
  \[ z = \frac{p-\pi_0}{\sqrt{\frac{\pi_0 (1-\pi_0)}{n}}} \sim \mathcal{N}(0,1) \]

  \begin{itemize}
  \tightlist
  \item
    \(\pi_0\) = hypothesized proportion
  \item
    \(p\) = sample proportion
  \end{itemize}
\end{itemize}

\hypertarget{t-test-for-2-independent-samples}{%
\subsection{t-test for 2 independent samples}\label{t-test-for-2-independent-samples}}

\begin{itemize}
\item
  Assumptions:

  \begin{itemize}
  \tightlist
  \item
    two independent samples are randomly selected from two populations with the same variance
  \item
    if you cannot use the assumption of same variance, use the Welch two-sample t-test

    \begin{itemize}
    \item
      test statistic is the same as below, but degrees of freedom are adjusted
    \item
    \end{itemize}
  \item
    if populations are not normally distributed, the sample sizes \(n_1\) and \(n_2\) from the two populations needs to be at least 30 to ensure that the distribution of the sample means are normal by the Central Limit Theorem
  \end{itemize}
\item
  Two-tailed Hypothesis:
  \[ H_0: \mu_1 = \mu_2 \]
  \[ H_1: \mu_1 \neq \mu_2 \]

  \begin{itemize}
  \tightlist
  \item
    \(\mu_1\) = population mean of 1st sample
  \item
    \(\mu_2\) = population mean of 2nd sample
  \end{itemize}
\item
  Test Statistic:
  \[ \frac{ (\overline{X}_1-\overline{X}_2) - (\mu_1 - \mu_2) }{ \sqrt{S_p^2 \left( \frac{1}{n_1} + \frac{1}{n_2} \right) } } \sim t_{(n_1 + n_2 -2)} \]
  \[ S_p^2 = \frac{(n_1-1)S_1^2 + (n_2-1)S_2^2}{(n_1-1)+(n_2-1)} \]

  \begin{itemize}
  \tightlist
  \item
    \(S_p\) = pooled variance
  \item
    \(\overline{X}_1\) = mean of 1st sample
  \item
    \(\overline{X}_2\) = mean of 2nd sample
  \item
    \(S_1^2\) = variance of 1st sample
  \item
    \(S_2^2\) = variance of 2nd sample
  \end{itemize}
\item
  \href{http://www.biostathandbook.com/twosamplettest.html}{More Info}
\item
  \href{https://rcompanion.org/rcompanion/d_02.html}{R Example}

  \begin{itemize}
  \item
    includes examples under both assumptions of equal and unequal variances
  \item
    Andrew Heiss provides a brief \href{https://www.andrewheiss.com/blog/2019/01/29/diff-means-half-dozen-ways/\#simulation-based-tests}{tutorial} with frequentist, simulation-based, and Bayesian approaches to comparing means between two groups. Also see Matti Vuorre's \href{https://vuorre.netlify.com/post/2017/01/02/how-to-compare-two-groups-with-robust-bayesian-estimation-using-r-stan-and-brms/\#equal-variances-t-test}{tutorial} for more details.
  \end{itemize}
\end{itemize}

\hypertarget{paired-t-test}{%
\subsection{paired t-test}\label{paired-t-test}}

\begin{itemize}
\item
  Assumptions:
\item
  \href{http://rcompanion.org/handbook/I_04.html}{More Info with R Example}
\end{itemize}

\hypertarget{chi-squared-test-for-proportions}{%
\subsection{chi-squared test for proportions}\label{chi-squared-test-for-proportions}}

\begin{itemize}
\tightlist
\item
  The chi-squared test for 2 x 2 frequency tables is equivalent to the square of the z-test for two proportions. See this \href{http://rinterested.github.io/statistics/chi_square_same_as_z_test.html}{link} for detailed explanation.
\end{itemize}

\hypertarget{chi-squared-test-for-independence}{%
\subsection{chi-squared test for independence}\label{chi-squared-test-for-independence}}

\begin{itemize}
\tightlist
\item
  Explain connection between chi-squared test for independence and log-linear models, which are Poisson models for categorical data.
\end{itemize}

\hypertarget{anova}{%
\subsection{ANOVA}\label{anova}}

\hypertarget{numerical-methods}{%
\section{Numerical Methods}\label{numerical-methods}}

In AP Calculus, you mostly encountered problems that can be solved analytically. However, in research, many differential equation models do not have analytical forms and must be solved numerically. Matlab is often used in applied math, engineering, and physical sciences for such cases as well as other modeling applications. Octave is an open-source alternative to Matlab. While R not the first language that comes to mind for numerical methods, many \href{https://cran.r-project.org/web/views/NumericalMathematics.html}{numerical R packages} have been developed as well as integration with Matlab, Octave, and Julia.

\begin{itemize}
\tightlist
\item
  \href{https://www.mathworks.com/moler/chapters.html}{Numerical Computing with Matlab}

  \begin{itemize}
  \tightlist
  \item
    This site has PDF versions of Cleve Moler's textbook on numerical computing alongside a \href{https://www.mathworks.com/academia/courseware/learn-differential-equations.html}{video series} with lectures on differential equations and linear algebra by Prof.~Gilbert Strang and computational video tutorials by Moler.
  \end{itemize}
\item
  \href{http://rstudio-pubs-static.s3.amazonaws.com/32888_197d1a1896534397b67fb04e0d4899ae.html}{Numerically Solving Differential Euqations with R}
\end{itemize}

\hypertarget{root-finding-algorithms}{%
\subsection{Root-Finding Algorithms}\label{root-finding-algorithms}}

\begin{itemize}
\tightlist
\item
  \href{https://rpubs.com/aaronsc32/newton-raphson-method}{Newton-Raphson Method Using R}
\item
  \href{https://rpubs.com/aaronsc32/bisection-method-r}{Bisection Method Using R}
\item
  \href{https://rpubs.com/aaronsc32/secant-method-r}{Secant Method Using R}
\end{itemize}

\hypertarget{numerical-solutions-to-differential-equations}{%
\subsection{Numerical Solutions to Differential Equations}\label{numerical-solutions-to-differential-equations}}

\begin{itemize}
\tightlist
\item
  \href{http://www.ohiouniversityfaculty.com/youngt/IntNumMeth/lecture30.pdf}{Euler Method Using Matlab}
\item
  Runge-Kutta Methods
\end{itemize}

\hypertarget{resources-by-discipline}{%
\chapter*{Resources by Discipline}\label{resources-by-discipline}}
\addcontentsline{toc}{chapter}{Resources by Discipline}

\hypertarget{biology-biostatistics}{%
\section*{Biology \& Biostatistics}\label{biology-biostatistics}}
\addcontentsline{toc}{section}{Biology \& Biostatistics}

\begin{itemize}
\tightlist
\item
  \href{http://www.biostathandbook.com/}{Handbook of Biological Statistics}
\item
  \href{https://rcompanion.org/rcompanion/index.html}{An R Companion for the Handbook of Biological Statistics}
\end{itemize}

\hypertarget{economics-econometrics}{%
\section*{Economics \& Econometrics}\label{economics-econometrics}}
\addcontentsline{toc}{section}{Economics \& Econometrics}

\begin{itemize}
\item
  \href{https://www.econometrics-with-r.org}{Introduction to Econometrics with R}
\item
  \href{https://bookdown.org/ccolonescu/RPoE4/}{Principles of Econometrics with R}
\item
  \href{https://bookdown.org/ronsarafian/IntrotoDS/}{Introduction to Data Science}
\item
  \href{http://www.urfie.net/read.html}{Using R for Introductory Econometrics}
\item
  Examples:

  \begin{itemize}
  \item
    \href{https://minerva.union.edu/dvorakt/43/sample_paper.htm}{Annotated Sample Econometrics Paper}
  \item
    \href{https://www.andrewheiss.com/blog/2019/02/16/algebra-calculus-r-yacas/}{Microeconomic example of utility maximization constrained by budget lines}
  \end{itemize}
\end{itemize}

\hypertarget{psychology}{%
\section*{Psychology}\label{psychology}}
\addcontentsline{toc}{section}{Psychology}

\begin{itemize}
\tightlist
\item
  \href{https://www.simplypsychology.org/qualitative-quantitative.html}{Psychology Research Methods}
\end{itemize}

\hypertarget{public-health-epidemiology}{%
\section*{Public Health \& Epidemiology}\label{public-health-epidemiology}}
\addcontentsline{toc}{section}{Public Health \& Epidemiology}

\begin{itemize}
\item
  Examples:

  \begin{itemize}
  \tightlist
  \item
    \href{https://rpubs.com/choisy/sir}{SIR Model Using R}
  \end{itemize}
\end{itemize}

\hypertarget{social-sciences}{%
\section*{Social Sciences}\label{social-sciences}}
\addcontentsline{toc}{section}{Social Sciences}

\begin{itemize}
\tightlist
\item
  \href{https://modu.ssri.duke.edu/}{Social Science Methods Modules}
\item
  \href{https://bookdown.org/paulcbauer/causal_analysis/}{Applied Causal Analysis}
\end{itemize}

\hypertarget{data}{%
\chapter{Data}\label{data}}

\hypertarget{data-sources-by-discipline}{%
\section{Data Sources by Discipline}\label{data-sources-by-discipline}}

\hypertarget{demography-and-official-statistics}{%
\subsection{Demography and Official Statistics}\label{demography-and-official-statistics}}

\begin{itemize}
\item
  \href{https://www.census.gov/data.html}{U.S. Census Data}

  \begin{itemize}
  \tightlist
  \item
    \href{https://factfinder.census.gov/faces/nav/jsf/pages/index.xhtml}{American Fact Finder}
  \item
    \href{https://usa.ipums.org/usa/}{IPUMS}

    \begin{itemize}
    \tightlist
    \item
      U.S. census microdata with social, economic, and health variables.
    \item
      Create custom data sets or use online tool.
    \end{itemize}
  \end{itemize}
\item
  \href{https://www.ons.gov.uk/peoplepopulationandcommunity}{UK Office for National Statistics}
\item
  \href{https://www.statcan.gc.ca/eng/start}{Statistics Canada}
\end{itemize}

\hypertarget{economics}{%
\subsection{Economics}\label{economics}}

\begin{itemize}
\tightlist
\item
  \href{https://psidonline.isr.umich.edu/}{Panel Study of Income Dynamics}
\item
  \href{http://www.sca.isr.umich.edu/}{University of Michigan Surveys of Consumers}
\end{itemize}

\hypertarget{education}{%
\subsection{Education}\label{education}}

\begin{itemize}
\tightlist
\item
  \href{https://ies.ed.gov/data.asp}{Institute of Education Sciences: Data Files}
\item
  \href{https://www.nationsreportcard.gov/ndecore/landing}{National Assessment of Educational Progress Data Explorer}
\end{itemize}

\hypertarget{law}{%
\subsection{Law}\label{law}}

\begin{itemize}
\tightlist
\item
  \href{https://case.law/}{Caselaw Access Project}

  \begin{itemize}
  \tightlist
  \item
    Digital access to U.S. state and federal cases from the 1600s to present.
  \end{itemize}
\end{itemize}

\hypertarget{social-sciences-1}{%
\subsection{Social Sciences}\label{social-sciences-1}}

\begin{itemize}
\tightlist
\item
  \href{https://www.icpsr.umich.edu/icpsrweb/ICPSR/}{ICPSR}
\end{itemize}

\hypertarget{data-documentation}{%
\section{Data Documentation}\label{data-documentation}}

Cite the source of your data. Provide links to the original data source and accompanying codebook, if any. Your data documentation will document your data analysis from the download of the raw data to the final steps of data analysis.

\begin{itemize}
\item
  \href{http://www.ddialliance.org/training/getting-started-new-content/create-a-codebook}{Create a Codebook}

  \begin{itemize}
  \tightlist
  \item
    List of codebook creation tools with guides and download links.
  \end{itemize}
\item
  \href{http://www.medicine.mcgill.ca/epidemiology/joseph/pbelisle/CodebookCookbook/CodebookCookbook.pdf}{Guide to Writing a Codebook}
\item
  \href{https://osf.io/zrsxd/download/}{How to Use R Codebook Package}
\end{itemize}

\hypertarget{analysis}{%
\chapter{Analysis}\label{analysis}}

\begin{itemize}
\item
  Logical Fallacies

  Read about the \href{https://medium.com/@pnhoward/12-common-fallacies-used-in-social-research-9713e4d9bf48}{common fallacies} in social research. Summary below:

  \begin{itemize}
  \tightlist
  \item
    fallacies of authority
  \item
    fallacies of logic
  \item
    fallacies of emotion
  \end{itemize}
\item
  Statistical Biases

  \begin{itemize}
  \tightlist
  \item
    Sampling bias

    \begin{itemize}
    \tightlist
    \item
      e.g., 1948 U.S. presidential election (see this \href{https://www.math.upenn.edu/~deturck/m170/wk4/lecture/case2.html}{case study})
    \item
      even very large samples could have sampling biases if sampling methods are poor and unrepresentative of the population (e.g., \href{https://www.math.upenn.edu/~deturck/m170/wk4/lecture/case1.html}{1936 Literary Digest Poll})
    \end{itemize}
  \item
    Omitted variable bias
  \item
    Nonresponse bias
  \item
    Selection bias
  \item
    Survivorship bias

    \begin{itemize}
    \tightlist
    \item
      e.g., when bankrupt companies are removed from a stock index and replaced with profitable companies, the index would experience an upward bias. Business failures would not be accounted for in time series data.
    \end{itemize}
  \end{itemize}
\end{itemize}

\hypertarget{data-programming}{%
\chapter*{Data Programming}\label{data-programming}}
\addcontentsline{toc}{chapter}{Data Programming}

\begin{itemize}
\item
  \href{https://www.r-project.org/}{R}

  \begin{itemize}
  \tightlist
  \item
    To download R, choose a \href{https://cran.r-project.org/mirrors.html}{CRAN mirror} closest to your geographic location.
  \item
    In order to build R packages, you should also download the latest recommended version of \href{https://cran.r-project.org/bin/windows/Rtools/}{Rtools}. Currently, the latest recommended version is \texttt{Rtoools35.exe}.
  \item
    During the installation of Rtools, you may need to add in \texttt{"C:\textbackslash{}Rtools\textbackslash{}mingw\_64\textbackslash{}bin;"} to the path.
  \end{itemize}
\item
  \href{https://www.rstudio.com/products/rstudio/download/\#download}{R Studio}

  \begin{itemize}
  \tightlist
  \item
    R Studio is an integrated development environment (IDE) for R. After downloading R Studio, you should be able to type the following command at the console to download some common R packages for data analysis and visualization.
  \end{itemize}
\end{itemize}

\begin{Shaded}
\begin{Highlighting}[]
\KeywordTok{install.packages}\NormalTok{(}\KeywordTok{c}\NormalTok{(}\StringTok{"dplyr"}\NormalTok{, }\StringTok{"tidyr"}\NormalTok{, }\StringTok{"ggplot2"}\NormalTok{, }\StringTok{"esquisse"}\NormalTok{, }\StringTok{"stats"}\NormalTok{, }\StringTok{"xtable"}\NormalTok{))}
\end{Highlighting}
\end{Shaded}

\hypertarget{cleaning-and-reshaping-data}{%
\section{Cleaning and Reshaping Data}\label{cleaning-and-reshaping-data}}

\begin{Shaded}
\begin{Highlighting}[]
\KeywordTok{library}\NormalTok{(reshape2)}
\KeywordTok{library}\NormalTok{(tidyr)}
\KeywordTok{library}\NormalTok{(xtable)}
\KeywordTok{library}\NormalTok{(stringr)}
\KeywordTok{library}\NormalTok{(knitr)}
\KeywordTok{options}\NormalTok{(}\DataTypeTok{kableExtra.latex.load_packages =} \OtherTok{FALSE}\NormalTok{)}
\KeywordTok{library}\NormalTok{(kableExtra)}
\KeywordTok{library}\NormalTok{(pander)}


\CommentTok{#original data is organized by id/trial (two locations per entry)}
\NormalTok{game <-}\StringTok{ }\KeywordTok{data.frame}\NormalTok{(}\DataTypeTok{id =} \KeywordTok{c}\NormalTok{(}\KeywordTok{rep}\NormalTok{(}\StringTok{"X"}\NormalTok{,}\DecValTok{3}\NormalTok{), }\KeywordTok{rep}\NormalTok{(}\StringTok{"Y"}\NormalTok{,}\DecValTok{3}\NormalTok{), }\KeywordTok{rep}\NormalTok{(}\StringTok{"Z"}\NormalTok{,}\DecValTok{3}\NormalTok{)),}
           \DataTypeTok{trial =} \KeywordTok{rep}\NormalTok{(}\KeywordTok{c}\NormalTok{(}\DecValTok{1}\NormalTok{,}\DecValTok{2}\NormalTok{,}\DecValTok{3}\NormalTok{), }\DecValTok{3}\NormalTok{),}
           \DataTypeTok{location_A =} \KeywordTok{round}\NormalTok{(}\KeywordTok{rnorm}\NormalTok{(}\DecValTok{9}\NormalTok{, }\DataTypeTok{mean =} \DecValTok{0}\NormalTok{, }\DataTypeTok{sd =} \DecValTok{1}\NormalTok{), }\DecValTok{1}\NormalTok{),}
           \DataTypeTok{location_B =} \KeywordTok{round}\NormalTok{(}\KeywordTok{rnorm}\NormalTok{(}\DecValTok{9}\NormalTok{, }\DataTypeTok{mean =} \DecValTok{0}\NormalTok{, }\DataTypeTok{sd =} \DecValTok{1}\NormalTok{), }\DecValTok{1}\NormalTok{))}

\CommentTok{# reshape data from wide to long (each entry is unique by id/trial/location)}
\NormalTok{game_long <-}\StringTok{ }\KeywordTok{melt}\NormalTok{(game, }\DataTypeTok{id =} \KeywordTok{c}\NormalTok{(}\StringTok{"id"}\NormalTok{,}\StringTok{"trial"}\NormalTok{), }\DataTypeTok{value.name =} \StringTok{"score"}\NormalTok{)}
\NormalTok{game_long}\OperatorTok{$}\NormalTok{variable <-}\StringTok{ }\KeywordTok{str_sub}\NormalTok{(game_long}\OperatorTok{$}\NormalTok{variable,}\OperatorTok{-}\DecValTok{1}\NormalTok{,}\OperatorTok{-}\DecValTok{1}\NormalTok{)}
\KeywordTok{colnames}\NormalTok{(game_long)[}\DecValTok{3}\NormalTok{] <-}\StringTok{ "location"}

\CommentTok{# reshape data back to wide (same as original data)}
\NormalTok{game_wide <-}\StringTok{ }\KeywordTok{dcast}\NormalTok{(game_long, id }\OperatorTok{+}\StringTok{ }\NormalTok{trial }\OperatorTok{~}\StringTok{ }\NormalTok{location, }\DataTypeTok{value.var =} \StringTok{"score"}\NormalTok{)}
\CommentTok{# reshape data into even wider form (one entry per id with 6 value columns: 2 locations X 3 trials)}
\NormalTok{game_wider <-}\StringTok{ }\KeywordTok{dcast}\NormalTok{(game_long, id }\OperatorTok{~}\StringTok{ }\NormalTok{location }\OperatorTok{+}\StringTok{ }\NormalTok{trial, }\DataTypeTok{value.var =} \StringTok{"score"}\NormalTok{)}

\CommentTok{# using tidyr and dplyr to reshape data}
\NormalTok{game_long2 <-}\StringTok{ }\NormalTok{game }\OperatorTok\StringTok{ }\KeywordTok{gather}\NormalTok{(label, score, location_A, location_B) }\OperatorTok
\StringTok{    }\KeywordTok{separate}\NormalTok{(label, }\KeywordTok{c}\NormalTok{(}\StringTok{"label_p1"}\NormalTok{,}\StringTok{"location"}\NormalTok{), }\DataTypeTok{sep =} \StringTok{"_"}\NormalTok{) }\OperatorTok
\StringTok{    }\NormalTok{dplyr}\OperatorTok{::}\KeywordTok{select}\NormalTok{(}\OperatorTok{-}\NormalTok{label_p1)}

\NormalTok{game_wide2 <-}\StringTok{ }\NormalTok{game_long2 }\OperatorTok\StringTok{ }\KeywordTok{spread}\NormalTok{(location, }\DataTypeTok{value =}\NormalTok{ score)}

\CommentTok{#unite() function creates the location X trial combinations first in long format # then apply the spread() function to reshape into wide format}
\CommentTok{#just like in game_wide, each entry in game_wide2 is unique by id}
\NormalTok{game_wider2 <-}\StringTok{ }\NormalTok{game_long2 }\OperatorTok\StringTok{ }\KeywordTok{unite}\NormalTok{(location_trial, location, trial) }\OperatorTok
\StringTok{    }\KeywordTok{spread}\NormalTok{(location_trial, }\DataTypeTok{value =}\NormalTok{ score)}

\CommentTok{#xtable method}
\CommentTok{#print(xtable(game, caption = "Wide Data Listed by Person/Trial (Scores by Location)"), type="html")}

\CommentTok{#kable method}
\CommentTok{#kable(game, caption = "Wide Data Listed by Person/Trial (Scores by Location)", booktabs = TRUE) %>%}
\CommentTok{#    kable_styling(latex_options = c("hold_position"))}

\CommentTok{#pander method (most flexible)}
\KeywordTok{pandoc.table}\NormalTok{(game, }\DataTypeTok{caption =} \StringTok{"(}\CharTok{\textbackslash{}\textbackslash{}}\StringTok{#tab:wide) Wide Data Listed by Person/Trial (Scores by Location)"}\NormalTok{)}
\end{Highlighting}
\end{Shaded}

\begin{longtable}[]{@{}cccc@{}}
\caption{\label{tab:wide} Wide Data Listed by Person/Trial (Scores by Location)}\tabularnewline
\toprule
\begin{minipage}[b]{0.06\columnwidth}\centering
id\strut
\end{minipage} & \begin{minipage}[b]{0.10\columnwidth}\centering
trial\strut
\end{minipage} & \begin{minipage}[b]{0.16\columnwidth}\centering
location\_A\strut
\end{minipage} & \begin{minipage}[b]{0.16\columnwidth}\centering
location\_B\strut
\end{minipage}\tabularnewline
\midrule
\endfirsthead
\toprule
\begin{minipage}[b]{0.06\columnwidth}\centering
id\strut
\end{minipage} & \begin{minipage}[b]{0.10\columnwidth}\centering
trial\strut
\end{minipage} & \begin{minipage}[b]{0.16\columnwidth}\centering
location\_A\strut
\end{minipage} & \begin{minipage}[b]{0.16\columnwidth}\centering
location\_B\strut
\end{minipage}\tabularnewline
\midrule
\endhead
\begin{minipage}[t]{0.06\columnwidth}\centering
X\strut
\end{minipage} & \begin{minipage}[t]{0.10\columnwidth}\centering
1\strut
\end{minipage} & \begin{minipage}[t]{0.16\columnwidth}\centering
0.5\strut
\end{minipage} & \begin{minipage}[t]{0.16\columnwidth}\centering
-1.5\strut
\end{minipage}\tabularnewline
\begin{minipage}[t]{0.06\columnwidth}\centering
X\strut
\end{minipage} & \begin{minipage}[t]{0.10\columnwidth}\centering
2\strut
\end{minipage} & \begin{minipage}[t]{0.16\columnwidth}\centering
0\strut
\end{minipage} & \begin{minipage}[t]{0.16\columnwidth}\centering
-1\strut
\end{minipage}\tabularnewline
\begin{minipage}[t]{0.06\columnwidth}\centering
X\strut
\end{minipage} & \begin{minipage}[t]{0.10\columnwidth}\centering
3\strut
\end{minipage} & \begin{minipage}[t]{0.16\columnwidth}\centering
0.6\strut
\end{minipage} & \begin{minipage}[t]{0.16\columnwidth}\centering
0.1\strut
\end{minipage}\tabularnewline
\begin{minipage}[t]{0.06\columnwidth}\centering
Y\strut
\end{minipage} & \begin{minipage}[t]{0.10\columnwidth}\centering
1\strut
\end{minipage} & \begin{minipage}[t]{0.16\columnwidth}\centering
-0.5\strut
\end{minipage} & \begin{minipage}[t]{0.16\columnwidth}\centering
0.2\strut
\end{minipage}\tabularnewline
\begin{minipage}[t]{0.06\columnwidth}\centering
Y\strut
\end{minipage} & \begin{minipage}[t]{0.10\columnwidth}\centering
2\strut
\end{minipage} & \begin{minipage}[t]{0.16\columnwidth}\centering
0.4\strut
\end{minipage} & \begin{minipage}[t]{0.16\columnwidth}\centering
0.7\strut
\end{minipage}\tabularnewline
\begin{minipage}[t]{0.06\columnwidth}\centering
Y\strut
\end{minipage} & \begin{minipage}[t]{0.10\columnwidth}\centering
3\strut
\end{minipage} & \begin{minipage}[t]{0.16\columnwidth}\centering
0.4\strut
\end{minipage} & \begin{minipage}[t]{0.16\columnwidth}\centering
1.5\strut
\end{minipage}\tabularnewline
\begin{minipage}[t]{0.06\columnwidth}\centering
Z\strut
\end{minipage} & \begin{minipage}[t]{0.10\columnwidth}\centering
1\strut
\end{minipage} & \begin{minipage}[t]{0.16\columnwidth}\centering
-1.4\strut
\end{minipage} & \begin{minipage}[t]{0.16\columnwidth}\centering
-1\strut
\end{minipage}\tabularnewline
\begin{minipage}[t]{0.06\columnwidth}\centering
Z\strut
\end{minipage} & \begin{minipage}[t]{0.10\columnwidth}\centering
2\strut
\end{minipage} & \begin{minipage}[t]{0.16\columnwidth}\centering
0.2\strut
\end{minipage} & \begin{minipage}[t]{0.16\columnwidth}\centering
-0.5\strut
\end{minipage}\tabularnewline
\begin{minipage}[t]{0.06\columnwidth}\centering
Z\strut
\end{minipage} & \begin{minipage}[t]{0.10\columnwidth}\centering
3\strut
\end{minipage} & \begin{minipage}[t]{0.16\columnwidth}\centering
-1.2\strut
\end{minipage} & \begin{minipage}[t]{0.16\columnwidth}\centering
0\strut
\end{minipage}\tabularnewline
\bottomrule
\end{longtable}

\begin{Shaded}
\begin{Highlighting}[]
\KeywordTok{pandoc.table}\NormalTok{(game_wider, }\DataTypeTok{caption =} \StringTok{"(}\CharTok{\textbackslash{}\textbackslash{}}\StringTok{#tab:wider) Wider Data Listed by ID (Scores by Location/Trial)"}\NormalTok{)}
\end{Highlighting}
\end{Shaded}

\begin{longtable}[]{@{}ccccccc@{}}
\caption{\label{tab:wider} Wider Data Listed by ID (Scores by Location/Trial)}\tabularnewline
\toprule
\begin{minipage}[b]{0.06\columnwidth}\centering
id\strut
\end{minipage} & \begin{minipage}[b]{0.08\columnwidth}\centering
A\_1\strut
\end{minipage} & \begin{minipage}[b]{0.07\columnwidth}\centering
A\_2\strut
\end{minipage} & \begin{minipage}[b]{0.08\columnwidth}\centering
A\_3\strut
\end{minipage} & \begin{minipage}[b]{0.08\columnwidth}\centering
B\_1\strut
\end{minipage} & \begin{minipage}[b]{0.08\columnwidth}\centering
B\_2\strut
\end{minipage} & \begin{minipage}[b]{0.08\columnwidth}\centering
B\_3\strut
\end{minipage}\tabularnewline
\midrule
\endfirsthead
\toprule
\begin{minipage}[b]{0.06\columnwidth}\centering
id\strut
\end{minipage} & \begin{minipage}[b]{0.08\columnwidth}\centering
A\_1\strut
\end{minipage} & \begin{minipage}[b]{0.07\columnwidth}\centering
A\_2\strut
\end{minipage} & \begin{minipage}[b]{0.08\columnwidth}\centering
A\_3\strut
\end{minipage} & \begin{minipage}[b]{0.08\columnwidth}\centering
B\_1\strut
\end{minipage} & \begin{minipage}[b]{0.08\columnwidth}\centering
B\_2\strut
\end{minipage} & \begin{minipage}[b]{0.08\columnwidth}\centering
B\_3\strut
\end{minipage}\tabularnewline
\midrule
\endhead
\begin{minipage}[t]{0.06\columnwidth}\centering
X\strut
\end{minipage} & \begin{minipage}[t]{0.08\columnwidth}\centering
0.5\strut
\end{minipage} & \begin{minipage}[t]{0.07\columnwidth}\centering
0\strut
\end{minipage} & \begin{minipage}[t]{0.08\columnwidth}\centering
0.6\strut
\end{minipage} & \begin{minipage}[t]{0.08\columnwidth}\centering
-1.5\strut
\end{minipage} & \begin{minipage}[t]{0.08\columnwidth}\centering
-1\strut
\end{minipage} & \begin{minipage}[t]{0.08\columnwidth}\centering
0.1\strut
\end{minipage}\tabularnewline
\begin{minipage}[t]{0.06\columnwidth}\centering
Y\strut
\end{minipage} & \begin{minipage}[t]{0.08\columnwidth}\centering
-0.5\strut
\end{minipage} & \begin{minipage}[t]{0.07\columnwidth}\centering
0.4\strut
\end{minipage} & \begin{minipage}[t]{0.08\columnwidth}\centering
0.4\strut
\end{minipage} & \begin{minipage}[t]{0.08\columnwidth}\centering
0.2\strut
\end{minipage} & \begin{minipage}[t]{0.08\columnwidth}\centering
0.7\strut
\end{minipage} & \begin{minipage}[t]{0.08\columnwidth}\centering
1.5\strut
\end{minipage}\tabularnewline
\begin{minipage}[t]{0.06\columnwidth}\centering
Z\strut
\end{minipage} & \begin{minipage}[t]{0.08\columnwidth}\centering
-1.4\strut
\end{minipage} & \begin{minipage}[t]{0.07\columnwidth}\centering
0.2\strut
\end{minipage} & \begin{minipage}[t]{0.08\columnwidth}\centering
-1.2\strut
\end{minipage} & \begin{minipage}[t]{0.08\columnwidth}\centering
-1\strut
\end{minipage} & \begin{minipage}[t]{0.08\columnwidth}\centering
-0.5\strut
\end{minipage} & \begin{minipage}[t]{0.08\columnwidth}\centering
0\strut
\end{minipage}\tabularnewline
\bottomrule
\end{longtable}

\begin{Shaded}
\begin{Highlighting}[]
\KeywordTok{pandoc.table}\NormalTok{(game_long, }\DataTypeTok{caption =} \StringTok{"(}\CharTok{\textbackslash{}\textbackslash{}}\StringTok{#tab:long) Long Data"}\NormalTok{)}
\end{Highlighting}
\end{Shaded}

\begin{longtable}[]{@{}cccc@{}}
\caption{\label{tab:long} Long Data}\tabularnewline
\toprule
\begin{minipage}[b]{0.06\columnwidth}\centering
id\strut
\end{minipage} & \begin{minipage}[b]{0.10\columnwidth}\centering
trial\strut
\end{minipage} & \begin{minipage}[b]{0.14\columnwidth}\centering
location\strut
\end{minipage} & \begin{minipage}[b]{0.10\columnwidth}\centering
score\strut
\end{minipage}\tabularnewline
\midrule
\endfirsthead
\toprule
\begin{minipage}[b]{0.06\columnwidth}\centering
id\strut
\end{minipage} & \begin{minipage}[b]{0.10\columnwidth}\centering
trial\strut
\end{minipage} & \begin{minipage}[b]{0.14\columnwidth}\centering
location\strut
\end{minipage} & \begin{minipage}[b]{0.10\columnwidth}\centering
score\strut
\end{minipage}\tabularnewline
\midrule
\endhead
\begin{minipage}[t]{0.06\columnwidth}\centering
X\strut
\end{minipage} & \begin{minipage}[t]{0.10\columnwidth}\centering
1\strut
\end{minipage} & \begin{minipage}[t]{0.14\columnwidth}\centering
A\strut
\end{minipage} & \begin{minipage}[t]{0.10\columnwidth}\centering
0.5\strut
\end{minipage}\tabularnewline
\begin{minipage}[t]{0.06\columnwidth}\centering
X\strut
\end{minipage} & \begin{minipage}[t]{0.10\columnwidth}\centering
2\strut
\end{minipage} & \begin{minipage}[t]{0.14\columnwidth}\centering
A\strut
\end{minipage} & \begin{minipage}[t]{0.10\columnwidth}\centering
0\strut
\end{minipage}\tabularnewline
\begin{minipage}[t]{0.06\columnwidth}\centering
X\strut
\end{minipage} & \begin{minipage}[t]{0.10\columnwidth}\centering
3\strut
\end{minipage} & \begin{minipage}[t]{0.14\columnwidth}\centering
A\strut
\end{minipage} & \begin{minipage}[t]{0.10\columnwidth}\centering
0.6\strut
\end{minipage}\tabularnewline
\begin{minipage}[t]{0.06\columnwidth}\centering
Y\strut
\end{minipage} & \begin{minipage}[t]{0.10\columnwidth}\centering
1\strut
\end{minipage} & \begin{minipage}[t]{0.14\columnwidth}\centering
A\strut
\end{minipage} & \begin{minipage}[t]{0.10\columnwidth}\centering
-0.5\strut
\end{minipage}\tabularnewline
\begin{minipage}[t]{0.06\columnwidth}\centering
Y\strut
\end{minipage} & \begin{minipage}[t]{0.10\columnwidth}\centering
2\strut
\end{minipage} & \begin{minipage}[t]{0.14\columnwidth}\centering
A\strut
\end{minipage} & \begin{minipage}[t]{0.10\columnwidth}\centering
0.4\strut
\end{minipage}\tabularnewline
\begin{minipage}[t]{0.06\columnwidth}\centering
Y\strut
\end{minipage} & \begin{minipage}[t]{0.10\columnwidth}\centering
3\strut
\end{minipage} & \begin{minipage}[t]{0.14\columnwidth}\centering
A\strut
\end{minipage} & \begin{minipage}[t]{0.10\columnwidth}\centering
0.4\strut
\end{minipage}\tabularnewline
\begin{minipage}[t]{0.06\columnwidth}\centering
Z\strut
\end{minipage} & \begin{minipage}[t]{0.10\columnwidth}\centering
1\strut
\end{minipage} & \begin{minipage}[t]{0.14\columnwidth}\centering
A\strut
\end{minipage} & \begin{minipage}[t]{0.10\columnwidth}\centering
-1.4\strut
\end{minipage}\tabularnewline
\begin{minipage}[t]{0.06\columnwidth}\centering
Z\strut
\end{minipage} & \begin{minipage}[t]{0.10\columnwidth}\centering
2\strut
\end{minipage} & \begin{minipage}[t]{0.14\columnwidth}\centering
A\strut
\end{minipage} & \begin{minipage}[t]{0.10\columnwidth}\centering
0.2\strut
\end{minipage}\tabularnewline
\begin{minipage}[t]{0.06\columnwidth}\centering
Z\strut
\end{minipage} & \begin{minipage}[t]{0.10\columnwidth}\centering
3\strut
\end{minipage} & \begin{minipage}[t]{0.14\columnwidth}\centering
A\strut
\end{minipage} & \begin{minipage}[t]{0.10\columnwidth}\centering
-1.2\strut
\end{minipage}\tabularnewline
\begin{minipage}[t]{0.06\columnwidth}\centering
X\strut
\end{minipage} & \begin{minipage}[t]{0.10\columnwidth}\centering
1\strut
\end{minipage} & \begin{minipage}[t]{0.14\columnwidth}\centering
B\strut
\end{minipage} & \begin{minipage}[t]{0.10\columnwidth}\centering
-1.5\strut
\end{minipage}\tabularnewline
\begin{minipage}[t]{0.06\columnwidth}\centering
X\strut
\end{minipage} & \begin{minipage}[t]{0.10\columnwidth}\centering
2\strut
\end{minipage} & \begin{minipage}[t]{0.14\columnwidth}\centering
B\strut
\end{minipage} & \begin{minipage}[t]{0.10\columnwidth}\centering
-1\strut
\end{minipage}\tabularnewline
\begin{minipage}[t]{0.06\columnwidth}\centering
X\strut
\end{minipage} & \begin{minipage}[t]{0.10\columnwidth}\centering
3\strut
\end{minipage} & \begin{minipage}[t]{0.14\columnwidth}\centering
B\strut
\end{minipage} & \begin{minipage}[t]{0.10\columnwidth}\centering
0.1\strut
\end{minipage}\tabularnewline
\begin{minipage}[t]{0.06\columnwidth}\centering
Y\strut
\end{minipage} & \begin{minipage}[t]{0.10\columnwidth}\centering
1\strut
\end{minipage} & \begin{minipage}[t]{0.14\columnwidth}\centering
B\strut
\end{minipage} & \begin{minipage}[t]{0.10\columnwidth}\centering
0.2\strut
\end{minipage}\tabularnewline
\begin{minipage}[t]{0.06\columnwidth}\centering
Y\strut
\end{minipage} & \begin{minipage}[t]{0.10\columnwidth}\centering
2\strut
\end{minipage} & \begin{minipage}[t]{0.14\columnwidth}\centering
B\strut
\end{minipage} & \begin{minipage}[t]{0.10\columnwidth}\centering
0.7\strut
\end{minipage}\tabularnewline
\begin{minipage}[t]{0.06\columnwidth}\centering
Y\strut
\end{minipage} & \begin{minipage}[t]{0.10\columnwidth}\centering
3\strut
\end{minipage} & \begin{minipage}[t]{0.14\columnwidth}\centering
B\strut
\end{minipage} & \begin{minipage}[t]{0.10\columnwidth}\centering
1.5\strut
\end{minipage}\tabularnewline
\begin{minipage}[t]{0.06\columnwidth}\centering
Z\strut
\end{minipage} & \begin{minipage}[t]{0.10\columnwidth}\centering
1\strut
\end{minipage} & \begin{minipage}[t]{0.14\columnwidth}\centering
B\strut
\end{minipage} & \begin{minipage}[t]{0.10\columnwidth}\centering
-1\strut
\end{minipage}\tabularnewline
\begin{minipage}[t]{0.06\columnwidth}\centering
Z\strut
\end{minipage} & \begin{minipage}[t]{0.10\columnwidth}\centering
2\strut
\end{minipage} & \begin{minipage}[t]{0.14\columnwidth}\centering
B\strut
\end{minipage} & \begin{minipage}[t]{0.10\columnwidth}\centering
-0.5\strut
\end{minipage}\tabularnewline
\begin{minipage}[t]{0.06\columnwidth}\centering
Z\strut
\end{minipage} & \begin{minipage}[t]{0.10\columnwidth}\centering
3\strut
\end{minipage} & \begin{minipage}[t]{0.14\columnwidth}\centering
B\strut
\end{minipage} & \begin{minipage}[t]{0.10\columnwidth}\centering
0\strut
\end{minipage}\tabularnewline
\bottomrule
\end{longtable}

\begin{itemize}
\tightlist
\item
  \href{https://www.rstudio.com/wp-content/uploads/2015/02/data-wrangling-cheatsheet.pdf}{Data Wrangling with dplyr and tidyr}
\end{itemize}

\hypertarget{regular-expressions}{%
\section{Regular Expressions}\label{regular-expressions}}

\begin{itemize}
\item
  \href{https://stringr.tidyverse.org/articles/regular-expressions.html}{Regular Expressions in R}
\item
  \href{https://www.rstudio.com/wp-content/uploads/2016/09/RegExCheatsheet.pdf}{Basic Regular Expressions in R Cheat Sheet}
\end{itemize}

\hypertarget{literate-programming}{%
\chapter*{Literate Programming}\label{literate-programming}}
\addcontentsline{toc}{chapter}{Literate Programming}

\hypertarget{latex}{%
\section{LaTeX}\label{latex}}

\begin{itemize}
\item
  \href{https://miktex.org/download}{MiKTeX}

  \begin{itemize}
  \tightlist
  \item
    First, download MiKTeX. Choose the version corresponding to your operating system (Windows, Mac, or Linux). Skip this step if you decide to use ShareLaTeX, which is an online LaTeX editor and does not require your computer to have underlying LaTeX packages via MiKTeX.
  \item
    Recommended, download the basic installer, which will download other uninstalled packages on the fly on an as-needed basis. If you want to download all packages, you can choose the Net Installer, but this may take up a lot of space.
  \end{itemize}
\item
  \href{https://blog.typeset.io/the-only-latex-editor-guide-youll-need-in-2018-e63868fae027}{Review of LaTeX Editors}

  \begin{itemize}
  \tightlist
  \item
    \href{https://www.overleaf.com/}{Overleaf/ShareLaTeX}
  \item
    \href{https://www.texstudio.org/}{TeXstudio}
  \item
    \href{https://www.lyx.org/}{LyX}
  \end{itemize}
\item
  \href{https://guides.library.harvard.edu/overleaf/latex}{LaTeX Guides}
\item
  \href{https://www.nyu.edu/projects/beber/files/Chang_LaTeX_sheet.pdf}{LaTeX Cheat Sheet}
\item
  Q and A:

  \begin{itemize}
  \tightlist
  \item
    \href{https://tex.stackexchange.com/questions/29172/link-to-file-in-the-parent-folder}{Reference File in Parent Folder}
  \end{itemize}
\end{itemize}

\hypertarget{beamer}{%
\section{Beamer}\label{beamer}}

Beamer is a LaTeX class for presentations.

\hypertarget{knitr-r-latex}{%
\section{knitr (R + LaTeX)}\label{knitr-r-latex}}

\begin{itemize}
\item
  \href{http://www.chrisbilder.com/stat850/LyXLaTeX/knitr/LyX-knitr1per.pdf}{Using knitr in LyX}
\item
  \href{https://www.pauljhurtado.com/teaching/software.html}{Configure Texstudio to use knitr}
\item
  \href{https://haozhu233.github.io/kableExtra/awesome_table_in_pdf.pdf}{Create LaTeX Tables with kable}

  \begin{itemize}
  \tightlist
  \item
    To avoid a incompatibility warning about the LaTeX \texttt{xcolor} package, place \texttt{options(kableExtra.latex.load\_packages\ =\ FALSE)} in your R chunk before \texttt{library(kableExtra)}. See Hao Zhu's explanation in page 4 of the link above.
  \end{itemize}
\item
  \href{http://haozhu233.github.io/kableExtra/}{kableExtra Vignettes}

  \begin{itemize}
  \tightlist
  \item
    vignettes for using outputting tables from R into HTML, LaTeX, and Word
  \end{itemize}
\item
  \href{https://rpubs.com/pankil/84526}{xtable and stargazer Examples}
\item
  \href{https://rapporter.github.io/pander/\#pander-an-r-pandoc-writer}{pander Tutorial}
\end{itemize}

\hypertarget{r-markdown}{%
\section{R Markdown}\label{r-markdown}}

\begin{itemize}
\tightlist
\item
  \href{https://github.com/adam-p/markdown-here/wiki/Markdown-Cheatsheet}{Markdown Reference}
\item
  \href{https://www.rstudio.com/wp-content/uploads/2016/03/rmarkdown-cheatsheet-2.0.pdf}{R Markdown Cheat Sheet}
\item
  \href{https://papers.ssrn.com/sol3/papers.cfm?abstract_id=3175518}{Writing a Reproducible Paper in R Markdown}
\end{itemize}

\hypertarget{r-bookdown}{%
\section{R Bookdown}\label{r-bookdown}}

\begin{itemize}
\item
  \href{https://bookdown.org/yihui/bookdown/}{Authoring Books with R Bookdown}
\item
  \href{https://bookdown.org/yihui/rmarkdown/}{R Markdown: The Definitive Guide}
\item
  \href{https://eddjberry.netlify.com/post/writing-your-thesis-with-bookdown/}{Writing Thesis with Bookdown}

  \begin{itemize}
  \tightlist
  \item
    Section on outputting into Microsoft Word using \texttt{bookdown::preview\_chapter()}
  \end{itemize}
\item
  \href{https://daijiang.name/en/2017/04/05/writing-academic-papers-with-rmarkdown/}{Writing Academic Papers with R Markdown}
\end{itemize}

\hypertarget{rmd-to-ms-word}{%
\section{Rmd to MS Word}\label{rmd-to-ms-word}}

\begin{itemize}
\item
  \href{https://rmarkdown.rstudio.com/articles_docx.html}{Rmd to docx}
\item
  \href{https://community.rstudio.com/t/nice-tables-when-knitting-to-word/3840}{Discussion on Using knitr for Word output}
\end{itemize}

\hypertarget{version-control}{%
\chapter*{Version Control}\label{version-control}}
\addcontentsline{toc}{chapter}{Version Control}

\begin{itemize}
\tightlist
\item
  \href{https://git-scm.com/downloads}{Git}
\end{itemize}

\hypertarget{github}{%
\section{Github}\label{github}}

\begin{itemize}
\tightlist
\item
  Create new repository
\end{itemize}

\begin{Shaded}
\begin{Highlighting}[]
\NormalTok{git init}
\NormalTok{git add README.md}
\NormalTok{git commit }\OperatorTok{-}\NormalTok{m }\StringTok{"first commit"}
\NormalTok{git remote add origin https}\OperatorTok{:}\ErrorTok{//}\NormalTok{github.com}\OperatorTok{/}\ErrorTok{<}\NormalTok{username}\OperatorTok{>}\ErrorTok{/<}\NormalTok{repo}\OperatorTok{-}\NormalTok{name}\OperatorTok{>}\NormalTok{.git}
\NormalTok{git push }\OperatorTok{-}\NormalTok{u origin master}
\end{Highlighting}
\end{Shaded}

\begin{itemize}
\item
  \href{http://r-pkgs.had.co.nz/git.html}{Best Practices Using Github in RStudio}
\item
  \href{https://vuorre.netlify.com/pdf/2017-Vuorre-Curley.pdf}{Tutorial on Git for Behavioral Sciences}
\item
  \href{https://happygitwithr.com/}{Github and R}
\end{itemize}

\bibliography{book.bib,packages.bib}


\end{document}
